\documentclass[en]{minipw} % wszystkie ustawienia szablonu są w minipw.cls; if in English, change [pl] to [en]
\allowdisplaybreaks
\usepackage{indentfirst}
\setlength{\parindent}{5mm} % wcięcie akapitowe 5mm, zarządzenie Rektora


% ------------ Ustawienia autora pracy ---------------

\setboolean{lady}{true} % kobiety wpisują true, mężczyźni - false

\title{English title} % nazwa pracy
\titleaux{Tytuł polski Tytuł polski Tytuł polski}
\type{magisters} % licencjat = licencjac, inżynier = inżyniers
\discipline{Matematyka} % kierunek
\specjal{specjalność}
\author{Bolesław Prus}
\album{100000}
\supervisor{dr~inż. Promotor Promotorski}
\konsultacje{prof. Dumbledore} % jeśli nie ma, trzeba zakomentować też w minipw.cls
\date{2018}
\klucze{slowo1, slowo2}
\keywords{k1, k2}
% ----------------------------------------------------

\begin{document}
\sloppy

% Nowy układ pracy dyplomowej

% 1. Strona tytułowa - trzeba wydrukować z osobnego pliku


% 2. Streszczenia
% Streszczenie ma zawierać tytuł pracy i słowa kluczowe
% if in English, English abstract goes first


\setcounter{page}{1}


\begin{streszczenie}

Lorem ipsum dolor sit amet, consetetur sadipscing elitr, sed diam nonumyeirmod tempor invidunt ut labore et dolore magna aliquyam erat, sed diamvoluptua. At vero eos et accusam et justo duo dolores et ea rebum. Stet clita kasd gubergren, no sea takimata sanctus est Lorem ipsum dolor sit amet.\\


\end{streszczenie}


\begin{abstract}

Lorem ipsum dolor sit amet, consetetur sadipscing elitr, sed diam nonumyeirmod tempor invidunt ut labore et dolore magna aliquyam erat, sed diamvoluptua. At vero eos et accusam et justo duo dolores et ea rebum. Stet clita kasd gubergren, no sea takimata sanctus est Lorem ipsum dolor sit amet.

Lorem ipsum dolor sit amet, consetetur sadipscing elitr, sed diam nonumyeirmod tempor invidunt ut labore et dolore magna aliquyam erat, sed diamvoluptua. At vero eos et accusam et justo duo dolores et ea rebum. Stet clita kasd gubergren, no sea takimata sanctus est Lorem ipsum dolor sit amet.\\

\end{abstract}

% 2. Oświadczenie o autorstwie pracy - w innym pliku
\makestatement


% 4. Spis treści
\cleardoublepage
\tableofcontents

% 5. Treść

\cleardoublepage
\pagestyle{fancy}

\chapter*{Wstęp}

O czym jest praca? Co się w niej znajduje? Jaki jest wkład autora?

Lorem ipsum dolor sit amet, consetetur sadipscing elitr, sed diam nonumyeirmod tempor invidunt ut labore et dolore magna aliquyam erat, sed diamvoluptua. At vero eos et accusam et justo duo dolores et ea rebum. Stet clita kasd gubergren, no sea takimata sanctus est Lorem ipsum dolor sit amet. Lorem ipsum dolor sit amet, consetetur sadipscing elitr, sed diam nonumyeirmod tempor invidunt ut labore et dolore magna aliquyam erat, sed diamvoluptua. At vero eos et accusam et justo duo dolores et ea rebum. Stet clita kasd gubergren, no sea takimata sanctus est Lorem ipsum dolor sit amet.


% --------------- PODSTAWOWE POJĘCIA ----------------
\chapter{Rozdział pokazowy}

Lorem ipsum dolor sit amet, consetetur sadipscing elitr, sed diam nonumyeirmod tempor invidunt ut labore et dolore magna aliquyam erat, sed diamvoluptua.

At vero eos et accusam et justo duo dolores et ea rebum.


\section{Przykładowa sekcja/podrozdział}

\begin{definition}[Definicja]
\textit{Definicją} nazywamy wypowiedź o określonej budowie, w której informuje się o znaczeniu pewnego wyrażenia przez wskazanie innego wyrażenia należącego do danego języka i posiadającego to samo znaczenie.
\end{definition}

\subsection{Podsekcja}

Poniżej podsekcji nie schodzimy.

\begin{definition}
\textit{Równaniem} nazywamy formę zdaniową postaci $t_1 = t_2$, gdzie $t_1, t_2$ są termami przynajmniej jeden z nich zawiera pewną zmienną.
\end{definition}

\begin{example}
Przykładem równania jest
\begin{equation}
2+2=4.
\end{equation}
Jeśli nie chcemy numerka, piszemy
\begin{equation*}
2+2=4.
\end{equation*}
Równanie (\ref{rownanie}) jest fałszywe. Referencje (i kilka innych rzeczy) działają po dwukrotnym przekompilowaniu tex-a.
\begin{equation}\label{rownanie}
\int \limits_{0}^{1} x \; dx = \frac{3}{2}.
\end{equation}

\end{example}

Twierdzenie \ref{Pitagoras} jest bardzo ciekawe.

\begin{theorem}[Twierdzenie Pitagorasa]\label{Pitagoras}
Niech będzie dany trójkąt prostokątny o przyprostokątnych długośći $a$ i $b$ oraz przeciwprostokątnej długości $c$. Wtedy
\[
a^2 + b^2 = c^2.
\]
\end{theorem}

\begin{proof}
Dowód został zaprezentowany w \cite{Ktos} oraz \cite{Innyktos}. Czyli w sumie mogę napisać, że w \cite{Ktos, Innyktos}. Albo że łatwo widać.
\end{proof}

\begin{corollary}
Doszedłem do jakiegoś wniosku i daję temu wyraz.
\end{corollary}




\begin{remark}
Lorem ipsum dolor sit amet, consetetur sadipscing elitr, sed diam nonumyeirmod tempor invidunt ut labore et dolore magna aliquyam erat, sed diamvoluptua. At vero eos et accusam et justo duo dolores et ea rebum.
\end{remark}

\begin{lemma}[Lemacik]
Ten lemat jest nie na temat.
\end{lemma}
\begin{proof} Dowód przez indukcję.
\end{proof}


Lorem ipsum dolor sit amet, consetetur sadipscing elitr, sed diam nonumyeirmod tempor invidunt ut labore et dolore magna aliquyam erat, sed diamvoluptua. At vero eos et accusam et justo duo dolores et ea rebum. Stet clita kasd gubergren, no sea takimata sanctus est Lorem ipsum dolor sit amet.Lorem ipsum dolor sit amet, consetetur sadipscing elitr, sed diam nonumyeirmod tempor invidunt ut labore et dolore magna aliquyam erat, sed diamvoluptua. At vero eos et accusam et justo duo dolores et ea rebum. Stet clita kasd gubergren, no sea takimata sanctus est Lorem ipsum dolor sit amet.



\section{Tabele i rysunki}

 Opcjonalny argument środowisk table i figure\\
h 	- 	bez przemieszczenia, dokładnie w miejscu użycia (uzyteczne w odniesieniu do niewielkich wstawek); \\
t 	- 	na górze strony;\\
b 	- 	na dole strony;\\
p 	- 	na stronie zawierającej wyłącznie wstawki;\\
! 	- 	ignorując większość parametrów kontrolujacych umieszczanie wstawek, przekroczenie wartosci, których może nie pozwolić na umieszczanie nastepnych wstawek na stronie.

\begin{figure}[h!]
\begin{center}
\setlength{\unitlength}{1mm}
\begin{picture}(40, 30)
\put(20,1){\line(0,1){20}} % linia

% dół
\put(20,1){\circle*{2}}
\put(25,1){0}

% góra
\put(20,21){\circle*{2}}
\put(25,21){1}
\end{picture}
\end{center}
\caption{Obrazek zrobiony w LaTeXu}
\end{figure}

\begin{table}[h!]
\centering
\begin{tabular}{rl|c}
bla & blabla & blablabla\\
\hline
bla & blabal & blablabla \\
ble & bleble & blebleble
\end{tabular}
\caption[Opis skrócony]{Pełny opis znajdujący się pod tabelą}
\end{table}

Lorem ipsum dolor sit amet, consetetur sadipscing elitr, sed diam nonumyeirmod tempor invidunt ut labore et dolore magna aliquyam erat, sed diamvoluptua. At vero eos et accusam et justo duo dolores et ea rebum. Stet clita kasd gubergren, no sea takimata sanctus est Lorem ipsum dolor sit amet.Lorem ipsum dolor sit amet, consetetur sadipscing elitr, sed diam nonumyeirmod tempor invidunt ut labore et dolore magna aliquyam erat, sed diamvoluptua. At vero eos et accusam et justo duo dolores et ea rebum. Stet clita kasd gubergren, no sea takimata sanctus est Lorem ipsum dolor sit amet.

\begin{figure}[h!]
\centering
\includegraphics[scale=0.5]{politechnika}
\caption[Logo MiNI]{Takie tam logo MiNI}
\end{figure}


\chapter{Następny rozdział}

A oto jakieś przykładowe drzewo wywodu zrobione przy pomocy pakietu forest.

\section{Jakiś podrozdział}


\begin{definition}
Niech $A\neq \emptyset$, $n \in \mathbb{N}$. Każde przekształcenie $f:A^n \rightarrow A$ nazywamy \textit{$n$-arną operacją} lub \textit{działaniem} określonym na $A$.
0-arne operacje to wyróżnione stałe.
\end{definition}


\begin{definition}[Algebra]
Parę uporządkowaną $(A,F)$, gdzie $A\neq \emptyset$ jest zbiorem, a $F$ jest rodziną operacji określonych na $A$, nazywamy \textit{algebrą} (lub \textit{$F$-algebrą}). Zbiór $A$ nazywa się \textit{zbiorem elementów}, \textit{nośnikiem} lub \textit{uniwersum} algebry $(A,F)$, a $F$ \textit{zbiorem operacji elementarnych}.
\end{definition}

\begin{proposition}
Stwierdzam więc ostatnio, że doszedłszy do granicy, pozostaje mi tylko przy tej granicy biwakować albo zawrócić, możliwie też szukać przejścia czy wyjścia na nowe obszary.
\end{proposition}



% 6. Bibliografia
% Bibliografia leksykograficznie wg nazwisk autorów

\begin{thebibliography}{20}%jak ktoś ma więcej książek, to niech wpisze większą liczbę
% \bibitem[numerek]{referencja} Autor, \emph{Tytuł}, Wydawnictwo, rok, strony
% cytowanie: \cite{referencja1, referencja 2,...}

\bibitem[1]{Ktos} A. Aaaaa, \emph{Tytuł}, Wydawnictwo, rok, strona-strona.
\bibitem[2]{Innyktos} J. Bobkowski, S. Dobkowski, \emph{Blebleble}, Magazyn nr, rok, strony.
\bibitem[3]{B} C. Brink, \emph{Power structures}, Algebra Universalis 30(2), 1993, 177-216.
\bibitem[4]{H} F. Burris, H. P. Sankappanavar, \emph{A Course of Universal Algebra}, Springer-Verlag, New York, 1981.

\end{thebibliography}



% 7. Wykaz symboli i skrótów - jeśli nie ma, zakomentować
\chapter*{Wykaz symboli i skrótów}

\begin{tabular}{cl}
nzw. & nadzwyczajny \\
* & operator gwiazdka \\
$\widetilde{}$ & tylda
\end{tabular}


% 8. Spis rysunków - jeśli nie ma, zakomentować (ale być może po prostu się nie zrobi)
\listoffigures


% 9. Spis tabel - jak wyżej
\renewcommand{\listtablename}{Spis tabel}
\listoftables


% 10. Spis załączników - jak nie ma załączników, to zakomentować lub usunąć
\chapter*{Spis załączników}
\begin{enumerate}
\item[1.] Załącznik 1
\item[2.] Załącznik 2
\end{enumerate}

% 11. Załączniki
\newpage
\pagestyle{empty}
Załącznik 1, załącznik 2 -- mają się znajdować na końcu pracy (to jest notka przypominająca)

\end{document}