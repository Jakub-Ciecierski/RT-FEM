\documentclass[en]{minipw} % wszystkie ustawienia szablonu są w minipw.cls; if in English, change [pl] to [en]
\allowdisplaybreaks
\usepackage{indentfirst}
\setlength{\parindent}{5mm} % wcięcie akapitowe 5mm, zarządzenie Rektora


% ------------ Ustawienia autora pracy ---------------

\setboolean{lady}{false} % kobiety wpisują true, mężczyźni - false

\title{Deformation of Soft Bodies on GPU} % nazwa pracy
\titleaux{Deformacja Ciał Odkształcalnych na GPU}
\type{magisters} % licencjat = licencjac, inżynier = inżyniers
\discipline{Informatyka} % kierunek
\specjal{Projektowanie Systemów CAD/CAM}
\author{Jakub Ciecierski}
\album{243260}
\supervisor{dr~inż. Joanna Porter}
%\konsultacje{prof. Dumbledore} % jeśli nie ma, trzeba zakomentować też w minipw.cls
\date{2018}
\klucze{Deformacje, MES, GPU}
\keywords{Deformations, FEM, GPU}
% ----------------------------------------------------

\begin{document}
\sloppy

% Nowy układ pracy dyplomowej

% 1. Strona tytułowa - trzeba wydrukować z osobnego pliku



% 2. Streszczenia
% Streszczenie ma zawierać tytuł pracy i słowa kluczowe
% if in English, English abstract goes first


\setcounter{page}{1}


\begin{abstract}

Abstract in English

\end{abstract}


\begin{streszczenie}

Abstract in Polish


\end{streszczenie}


% 3. Oświadczenie o autorstwie pracy - w innym pliku
\makestatement


% 4. Spis treści
\cleardoublepage
\tableofcontents

% 5. Treść

\cleardoublepage
\pagestyle{fancy}

\chapter*{Introduction}

\chapter{Background}

\section{Deformation}
\section{Rigid Body Dynamics}
\section{Collision}

\chapter{Finite Element Method}

\section{Linear FEM}
\section{Strong Formulation}


% 6. Bibliografia
% Bibliografia leksykograficznie wg nazwisk autorów

\begin{thebibliography}{20}%jak ktoś ma więcej książek, to niech wpisze większą liczbę
% \bibitem[numerek]{referencja} Autor, \emph{Tytuł}, Wydawnictwo, rok, strony
% cytowanie: \cite{referencja1, referencja 2,...}

\end{thebibliography}



% 7. Wykaz symboli i skrótów - jeśli nie ma, zakomentować

%\chapter*{Wykaz symboli i skrótów}
%\begin{tabular}{cl}
%nzw. & nadzwyczajny \\
%* & operator gwiazdka \\
%$\widetilde{}$ & tylda
%\end{tabular}


% 8. Spis rysunków - jeśli nie ma, zakomentować (ale być może po prostu się nie zrobi)
%\listoffigures


% 9. Spis tabel - jak wyżej
%\renewcommand{\listtablename}{Spis tabel}
%\listoftables


% 10. Spis załączników - jak nie ma załączników, to zakomentować lub usunąć
%\chapter*{Spis załączników}
%\begin{enumerate}
%\item[1.] Załącznik 1
%\item[2.] Załącznik 2
%\end{enumerate}

% 11. Załączniki
%\newpage
%\pagestyle{empty}
%Załącznik 1, załącznik 2 -- mają się znajdować na końcu pracy (to jest notka przypominająca)

\end{document}