\documentclass[12pt,twoside,a4paper]{article}
% ten dokument należy kompilować za pomocą XeLaTeX-a

\usepackage{fontspec,newunicodechar}
\defaultfontfeatures{Ligatures=TeX}
\usepackage[small,sf,bf]{titlesec}

\usepackage{graphicx}
\usepackage{geometry}
\usepackage{a4wide}
\geometry{left=20mm,right=20mm,bindingoffset=10mm, top=25mm, bottom=25mm}

\newfontfamily\arial{arial.ttf}

\linespread{1.5}
\setlength{\parindent}{0mm}

% ---------------------- Do wypełnienia ------------------------------

\newcommand{\discipline}{Matematyka}
\newcommand{\spec}{Matematyka w naukach informacyjnych}
\renewcommand{\title}{Tytuł pracy dyplomowej}
\renewcommand{\author}{Imię Nazwisko}
\newcommand{\supervisor}{stopień naukowy Imię Nazwisko}
\newcommand{\album}{123456}
\renewcommand{\year}{2017}

\begin{document}
\pagestyle{empty}

\begin{center}
\includegraphics[scale=1.]{politechnika} 
\vspace{70pt}


\includegraphics[scale=1.]{praca_lic} % lub praca_lic lub praca_inz lub praca_mgr

{ \arial na kierunku \discipline

% tylko dla pracy magisterkiej należy odkomentować poniższą linijkę

% \arial w specjalności \spec

\vspace{40pt}
{\arial \large \title}

\vspace{50pt}

{\arial \huge \author}

\vspace{5pt}

Numer albumu \album

\vspace{40pt}

promotor \\
{\arial \supervisor}

\vspace{15pt}
 
konsultacje (opcjonalnie)  \\
{\arial tyt./st. naukowy Imię Nazwisko }

 \vfill
WARSZAWA \year \\
}
\end{center}


\newpage
\null

\vfill

\begin{center}
\begin{tabular}[t]{ccc}
............................................. & \hspace*{100pt} & .............................................\\
podpis promotora & \hspace*{100pt} & podpis autora
\end{tabular}
\end{center}


\end{document}